\newpage
\section{Grundlagen Halbleiter}
	\begin{minipage}{8cm}
		\begin{align*}
			R &= \frac{\rho \cdot l}{A} = \frac{l}{\kappa \cdot A} \\
			\kappa &= \frac{1}{\rho}  = n \cdot q \cdot \mu \\
			\kappa &= n_i \cdot q \cdot \left( \mu_n - \mu_p \right) \\
			n_i^2 &= n_0 \cdot p_0 \\
			\rho(T) &= \rho(T_0) \cdot \left(1 + \alpha(T-T_0)\right)
		\end{align*}
		
	\end{minipage}
	\begin{minipage}{8cm}
		\begin{tabular}{lll}
		$\kappa$	& $[{m}/{\Omega mm^2}]$	& spezifische Leitfähigkeit \\
		$\rho$		& $[{\Omega mm^2}/{m}]$	& spezifischer Widerstand \\
		$n$			& $[cm^{-3}]$ & Ladungsträgerdichte \\
		$n_i$		& $[cm^{-3}]$ & Eigenleitungsdichte \\
		$n_0$		& $[cm^{-3}]$ & Elektronendichte \\
		$p_0$		& $[cm^{-3}]$ & Löcher-Dichte \\
		$q$			& $[As]$	& Elementarladung $q=1,6 \cdot 10^{-19} As$ \\ 
		$\mu$		& $[{cm^2}/{Vs}]$	& Beweglichkeit der Ladungsträger \\
		$l$			& $[m]$	& Länge \\
		$A$			& $[mm^2]$	& Fläche \\
		$\alpha$	& $[K^{-1}]$ & Temperaturkoeffizient \\
		$T$			& $[K]$	& Temperatur \\
		\end{tabular}
	\end{minipage}

\begin{multicols}{2}
	\subsection{Silizium}
	Si-Atom hat 4 Valenzelektronen auf äusserster Schale. Energetisch optimal wäre eine voll besetzte Schale mit 8 Valenzelektronen. Deshalb bildet Silizium ein Kristallgitter, in dem sich zwei benachbarte Si-Atome je ein Valenzelektron. 
	
	\subsection{Dotierung}
		N-Dotiert: höherwertiges Atom mit 5 VE (Donator)\\
		P-Dotiert: tieferwertiges Atom mit 3 VE (Akzeptor) \\
		
		Starke Dotierung: N$^+$, P$^+$ \\
		Schwache Dotierung: N$'-$, P$^-$ \\