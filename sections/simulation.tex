\section{Simulation}
	\subsection{Simulationsarten}
		{\bf DC-Analyse}
		\begin{itemize}
			\item Für Kennlinien und Arbeitspunkt
			\item Verhalten bei zeitlich konstanten Eingangsstimuli
			\item Kondensatoren sind offen, Spulen Kurzschlüsse
			\item Optimalerweise von jedem Knoten einen DC-Pfad zum Nullknoten
		\end{itemize}
		{\bf AC-Analyse}
		\begin{itemize}
			\item Kleinsignal-Übertragungsfunktion
			\item Verhalten bei sinusförmigen Eingangsstimuli
			\item Bei einem festen DC-Arbeitspunkt
			\item Nichtlineare Funktionen werden im Arbeitspunkt linearisiert
			\item Für Frequenzgänge
		\end{itemize}
		{\bf Transienten-Analyse}
		\begin{itemize}
			\item Zeitlicher Verlauf
			\item Verhalten bei beliebigen zeitlichen Eingangsstimuli
			\item lineares und nichtlineares Verhalten
			\item Bsp: Simulation des Einschwingverhaltens von Filtern
		\end{itemize}

